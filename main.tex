\documentclass{iwi}
\usepackage{paper}

\addbibresource{Literatur.bib}

\title{Entschlüsselung von Funksprüchen am Beispiel von Enigma}
\author{Alan Turing}

\begin{document}

\pagenumbering{Alph}
\begin{titlepage}{
	\sffamily 
	\begin{flushleft}
	\begin{doublespace}
		Universität Leipzig 						\\
		Wirtschaftswissenschaftliche Fakultät				\\
		Institut für Wirtschaftsinformatik				\\
		Professur für Wirtschaftsinformatik, insbesondere Softwareentwicklung für Wirtschaft und Verwaltung	\\
		Prof. Dr. Ulrich W. Eisenecker					\\
		David Baum, M. Sc						\\\ \\\ \\\ \\
	\end{doublespace}
	\end{flushleft}
	\begin{center}
		\begin{large}Thema\end{large}\\\ \\
		\begin{Large}
			\makeatletter
			\textbf{\@title}
			\makeatother
		\end{Large}\\\ \\\ \\\ \\
		
		Bachelorarbeit zur Erlangung des akademischen Grades \\
		Bachelor of Science – Wirtschaftsinformatik

		\vfill
	\end{center}
	\begin{onehalfspace}
	\begin{tabular}{ll}
			vorgelegt von: 	& Alan Turing 		\\
			Matrikelnummer:	& 1234567 		\\
			Email-Adresse: 	& turing@studserv.uni-leipzig.de	\\
			Telefonnummer: 	& +49 221 97 58 18 70	\\
			Anschrift:     	& Baker Street 221B\\
					& 04177 Leipzig	\\
			Leipzig, den 04. Mai 2077
	\end{tabular}\\\ \\\ \
	\end{onehalfspace}
}\end{titlepage} 

\include{Abstract}

\phantomsection 	% Damit Nummierung im Inhaltsverzeichnis und PDF-Lesezeichen richtig ist
\pagenumbering{Roman}
	
\tableofcontents 	% Inhaltsverzeichnis
\listoffigures 		% Abbildungsverzeichnis
\listoftables 		% Tabellenverzeichnis
\lstlistoflistings	% Listingverzeichnis
\newacronym[firstplural={Software-Produktlinien (SPL)}]{SPL}{SPL}{Software-Produktlinie}
\newacronym{Abk}{Abk}{Abk\"urzung} % hier keine Umlaute verwenden

\printglossary[type=\acronymtype, title = Abkürzungsverzeichnis, toctitle = Abkürzungsverzeichnis]





% \listoftodos % auskommentieren, wenn leer?

% Hauptteil

\newpage
\setcounter{frontpages}{\value{page}}
\pagenumbering{arabic}
\onehalfspacing % Eigentlich überflüssig, warum trotzdem notwendig?

\chapter{Einleitung} \label{chap:Einleitung}

\section{Motivation} \label{sec:Motivation}

\section{Problemstellung} \label{sec:Problemstellung}

\section{Aufbau der Arbeit} \label{sec:Aufbau}

\chapter{Beispiele} \label{chap:Beispiele}

\section{Ein Abschnitt} \label{sec:Abschnitt}

\subsection{Ein Unterabschnitt} \label{sec:Unterabschnitt}

Jeder Satz steht auf einer neuen Zeile.
Das merkt man im PDF gar nicht, vereinfacht aber das Arbeiten mit git, da git zeilenweise arbeitet.
Das ist eine Referenz zu Abschnitt~\ref{chap:Einleitung} mit einem geschützten Leerzeichen.

\section{Abkürzungen} \label{sec:Abkürzungen}

So verwendet man eine \gls{Abk} . Und zwar jedes Mal, wenn man \gls{Abk} benutzt. 
Latex überprüft automatisch, an welcher Stelle die \gls{Abk} das erste Mal verwendet wird und führt dort die \gls{Abk} ein.
In der Abkuerzungsverzeichnis.tex können beliebige eigene Abkürzungen definiert werden, zum Beispiel \gls{BEANS}.

\section{Abbildungen} \label{sec:Abbildungen}

Siehe Abbildung~\ref{img:meme}. Abbildungen werden unten oder oben auf einer Seite platziert. 
Die kurze Bildunterschrift taucht nur im Abbildungsverzeichnis auf.
Diese sollte prägnant sein und keine Quellenverweise enthalten.
Die vollständige Bildunterschrift steht unter der Abbildung und sollte die Quellenverweise sowie gegebenenfalls die Legende enthalten.

\begin{figure}
    \centering
    \includegraphics[width = 0.62\textwidth]{img/science.jpg}
    \caption[Das ist ein kurzer Untertitel]{Das ist ein langer Untertitel, der unter der Abbildung steht.}
    \label{img:meme}
\end{figure}

\section{Listings} \label{sec:Listings}

Siehe Listing~\ref{lst:helloWorld}. Auch Listings werden unten oder oben auf einer Seite platziert. 
Angabe der Sparche kann entsprechend geändert werden, das Syntax-Highlighting muss in paper.sty für jede Sprache einzeln festgelegt werden.


\begin{lstlisting}[float, caption={Hello, World!}, label={lst:helloWorld}, language=C++] 
#include <iostream>

int main() 
{
    std::cout << "Hello, World!" << std::endl;
    return 0;
}
\end{lstlisting}

\section{Tabellen} \label{sec:Tabellen}

Siehe Tabelle~\ref{tbl:pokemon} und Quelltextkommentare.

\begin{table}
\begin{center}
\begin{tabular}{c|ll} % c für eine zentrierte Spalte (#), l für eine links-ausgerichtete Spalte (Name) und das zweite l für die Spalte Typ
	\# & Name & Typ	\\
	\hline   
	1      & Bisasam   & Pflanze   \\
	137    & Porygon   & Normal    \\
	151    & Mew       & Psycho    \\

\end{tabular}
\caption{Eine Beispieltabelle}
\label{tbl:pokemon}
\end{center}
\end{table}

\section{Quellen}

Zum Angaben von Quellen gibt es zwei eigens definierte Befehle: $\backslash quelle$ und $\backslash zitat$.
$\backslash zitat$ ist für direkte Zitate. Kurze Zitate werden im Fließtext eingebunden, längere Zitate werden automatisch als eigener Absatz eingerückt.
$\backslash quelle$ ist für indirekte Zitate. 
Im Fließtext wird ``vgl'' automatisch erägnzt, innerhalb einer Abbildungsunterschrift ``in Anlehnung an''. 
Beide Befehle sollten für eine/n Autor/in, für zwei Autor/innen sowie für mehrere Autor/innen funktionieren.
Die Angabe der Seitenzahl erfolgt in eckigen Klammern. ``S.'' wird nicht verwendet, aber ``f'' bzw. ``ff'' wenn sich die Quellenangabe auch auf die nächste bzw. die nächsten Seiten bezieht.

Um die einzelnen Parameter zu verstehen, am besten direkt in den Quelltext schauen. Use the source, Luke! \quelle[3]{Lucas}
Mit Quellenangaben sollte immer sehr sorgfältig umgegangen werden. 
\zitat{Spiderman}{
With great power comes great responsibility}
Und hier noch ein langes Zitat, das automatisch eingerückt wird:
 \zitat[14f]{Carroll}{
Verdaustig war's, und glaße Wieben \\
rotterten gorkicht im Gemank. \\
Gar elump war der Pluckerwank, \\
und die gabben Schweisel frieben. \\

»Hab acht vorm Zipferlak, mein Kind! \\
Sein Maul ist beiß, sein Griff ist bohr. \\
Vorm Fliegelflagel sieh dich vor, \\
dem mampfen Schnatterrind.« \\

Er zückt' sein scharfgebifftes Schwert, \\
den Feind zu futzen ohne Saum, \\
und lehnt' sich an den Dudelbaum \\
und stand da lang in sich gekehrt. \\

In sich gekeimt, so stand er hier, \\
da kam verschnoff der Zipferlak \\
mit Flammenlefze angewackt \\
und gurgt' in seiner Gier. \\

Mit Eins! und Zwei! und bis auf's Bein! \\
Die biffe Klinge ritscheropf! \\
Trennt' er vom Hals den toten Kopf, \\
und wichernd sprengt' er heim. \\

»Vom Zipferlak hast uns befreit? \\
Komm an mein Herz, aromer Sohn! \\
Oh, blumer Tag! Oh, schlusse Fron!« \\
So kröpfte er vor Freud'. \\

Verdaustig war's, und glaße Wieben \\
rotterten gorkicht im Gemank. \\
Gar elump war der Pluckerwank, \\
und die gabben Schweisel frieben.
}

\section{Todo-Notes}

Es ist auch möglich, Notizen zu erstellen:

\todo[inline]{Silbentrennung erklären}

\section{Allgemeine Hinweise zu \LaTeX}

Jeder Satz sollte in einer eigenen Zeile stehen, sowie in diesem Dokument.
Das erleichtert das Arbeiten mit git, da git zeilenbasiert arbeitet und so die erstellen diffs viel kleiner werden.
Der Satz kann ganz normal mit einem Punkt beendet werden, \LaTeX fügt automatisch ein Leerzeichen ein.

Das ist ein neuer Absatz.
Er wurde einfach durch eine Leerzeile im \LaTeX-Dokument erzeugt.
Absätze sollten nicht manuell erzeugt, wie durch die Verwendung von \lstinline!\\!  oder anderen Methoden.
Manuelle Zeilenumbrüche sind in der Regel ebenfalls überflüssig und sollten nur in Ausnahmefällen, wie dem obigen Gedicht, verwendet werden.

% Schluss

 	\appendix
	\pagenumbering{Roman}
	\setcounter{page}{\value{frontpages}}
	\chapter{Die Großen Alten}

\begin{itemize}
\item Azathoth
\item Cthulhu
\item Ghatanothoa
\item Hastur
\item Nyarlathotep
\item Rhan-Tegoth
\item Shub-Niggurath
\item Tsathoggua
\item Yig
\item Yog-Sothoth
\end{itemize}

\chapter{Yet another appendix}

Hier könnte ihr Anhang stehen.


	\printbibliography[heading=bibintoc]

	\chapter*{Ehrenwörtliche Erklärung}
\thispagestyle{empty}
Ich versichere, dass ich die Bachelorarbeit selbstständig verfasst und keine anderen als die angegebenen Quellen und Hilfsmittel benutzt habe. 
Darüber hinaus versichere ich, dass die elektronische Version der Bachelorarbeit mit der gedruckten Version übereinstimmt.

\end{document}
