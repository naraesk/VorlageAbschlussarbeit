\documentclass{iwi}
\usepackage{paper}

\title{Entschlüsselung von Funksprüchen am Beispiel von Enigma}
\author{Alan Turing}

\begin{document}

\pagenumbering{Alph}
\begin{titlepage}{
	\sffamily 
	\begin{flushleft}
	\begin{doublespace}
		Universität Leipzig 						\\
		Wirtschaftswissenschaftliche Fakultät				\\
		Institut für Wirtschaftsinformatik				\\
		Professur für Wirtschaftsinformatik, insbesondere Softwareentwicklung für Wirtschaft und Verwaltung	\\
		Prof. Dr. Ulrich W. Eisenecker					\\
		David Baum, M. Sc						\\\ \\\ \\\ \\
	\end{doublespace}
	\end{flushleft}
	\begin{center}
		\begin{large}Thema\end{large}\\\ \\
		\begin{Large}
			\makeatletter
			\textbf{\@title}
			\makeatother
		\end{Large}\\\ \\\ \\\ \\
		
		Bachelorarbeit zur Erlangung des akademischen Grades \\
		Bachelor of Science – Wirtschaftsinformatik

		\vfill
	\end{center}
	\begin{onehalfspace}
	\begin{tabular}{ll}
			vorgelegt von: 	& Alan Turing 		\\
			Matrikelnummer:	& 1234567 		\\
			Email-Adresse: 	& turing@studserv.uni-leipzig.de	\\
			Telefonnummer: 	& +49 221 97 58 18 70	\\
			Anschrift:     	& Baker Street 221B\\
					& 04177 Leipzig	\\
			Leipzig, den 04. Mai 2077
	\end{tabular}\\\ \\\ \
	\end{onehalfspace}
}\end{titlepage} 

\include{Abstract}

\phantomsection 	% Damit Nummierung im Inhaltsverzeichnis und PDF-Lesezeichen richtig ist
\pagenumbering{Roman}
	
\tableofcontents 	% Inhaltsverzeichnis
\listoffigures 		% Abbildungsverzeichnis
\listoftables 		% Inhaltsverzeichnis
\lstlistoflistings	% Listingverzeichnis
\newacronym[firstplural={Software-Produktlinien (SPL)}]{SPL}{SPL}{Software-Produktlinie}
\newacronym{Abk}{Abk}{Abk\"urzung} % hier keine Umlaute verwenden

\printglossary[type=\acronymtype, title = Abkürzungsverzeichnis, toctitle = Abkürzungsverzeichnis]



% \listoftodos % auskommentieren, wenn leer?

% Hauptteil

\newpage
\setcounter{frontpages}{\value{page}}
\pagenumbering{arabic}
\onehalfspacing % Eigentlich überflüssig, warum trotzdem notwendig?

\chapter{Einleitung} \label{chap:Einleitung}

\section{Motivation} \label{sec:Motivation}

\section{Problemstellung} \label{sec:Problemstellung}

\section{Aufbau der Arbeit} \label{sec:Aufbau}

\chapter{Beispiele} \label{chap:Beispiele}

\section{Ein Abschnitt} \label{sec:Abschnitt}

\subsection{Ein Unterabschnitt} \label{sec:Unterabschnitt}

Jeder Satz steht auf einer neuen Zeile.
Das merkt man im PDF gar nicht, vereinfacht aber das Arbeiten mit git, da git zeilenweise arbeitet.
Das ist eine Referenz zu Abschnitt~\ref{chap:Einleitung} mit einem geschützten Leerzeichen.

\section{Abkürzungen} \label{sec:Abkürzungen}

So verwendet man eine \gls{Abk}. Und zwar jedes Mal, wenn man \gls{Abk} benutzt.

\section{Abbildungen} \label{sec:Abbildungen}

Siehe Abbildung~\ref{img:meme}. Abbildungen werden unten oder oben auf einer Seite platziert.

\begin{figure}
    \centering
    \includegraphics[width = 0.62\textwidth]{img/science.jpg}
    \caption{Das ist ein Untertitel}
    \label{img:meme}
\end{figure}

\section{Listings} \label{sec:Listings}

Siehe Listing~\ref{lst:helloWorld}. Auch Listings werden unten oder oben auf einer Seite platziert.
Angabe der Sparche kann entsprechend geändert werden, das Syntax-Highlighting muss in paper.sty für jede Sprache einzeln festgelegt werden.


\begin{lstlisting}[float, caption={Hello, World!}, label={lst:helloWorld}, language=C++] 
#include <iostream>

int main() 
{
    std::cout << "Hello, World!" << std::endl;
    return 0;
}
\end{lstlisting}

\section{Tabellen} \label{sec:Tabellen}

Siehe Tabelle~\ref{tbl:pokemon} und Quelltextkommentare.

\begin{table}
\begin{center}
\begin{tabular}{c|ll} % c für eine zentrierte Spalte (#), l für eine links-ausgerichtete Spalte (Name) und das zweite l für die Spalte Typ
	\# & Name & Typ	\\
	\hline   
	1      & Bisasam   & Pflanze   \\
	137    & Porygon   & Normal    \\
	151    & Mew       & Psycho    \\

\end{tabular}
\caption{Eine Beispieltabelle}
\label{tbl:pokemon}
\end{center}
\end{table}

\section{Quellen}

Mit dem Befehl todo können Notizen erstellt werden.

\todo[inline]{Quellen und Literaturverzeichnis}


\end{document}
